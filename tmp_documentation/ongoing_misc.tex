%&latex
\documentclass{article}

\begin{document}

%+Title
\title{Article Title}
\author{Author}
\date{\today}
\maketitle
%-Title



\section{Embedding in GF}

linear map $\sigma: GF(2^{k}) \rightarrow GF(2^{k_1}), $

first we have an integer of $k$ bits i.e. $(\alpha_{k-1}, \cdots, \alpha_{0})$
representing the following polynomials $\alpha_{k-1} x^{k-1} + \cdots + \alpha_{0}$, then we try to find a subfield of the larger field which is isomorphic to the smaller field. 

We factorize the reduction polynomial of finite field $GF(2^k)$ in the larger field $GF(2^{k_1})$ and take any root $e$. Then we derive the $k$th powers of the root, namely $(e^0, e^1, \cdots, e^{k-1})$. 

For any element $base = (\alpha_{k-1}, \cdots, \alpha_{0}) \in GF(2^k)$, the lifting element in $GF(2^{k_1})$
is 



r as the root of reduction polynomial f of GF2k

$\sigma(r) = e$

$f(r) = a_{k-1}\cdot r^{k-1} + \cdots + a_0 = 0$
 
$\sigma(base) = \sigma( a_{k-1}\cdot x^{k-1} + \cdots + a_0) 
= \sigma(a_{k-1}\cdot x^{k-1}) + \cdots + \sigma(a_{0}})
= a_{k-1}\cdot \sigma(x^{k-1}) + \cdots + a_{0} \cdot \sigma(1)}
= a_{k-1}\cdot e^{k-1} + \cdots + a_{0} \cdot e^{0} } $

Okay. 
$$ F 
= \sigma(\alpha_{k-1}, \cdots, \alpha_{0})
= \alpha_{k-1} \cdot e^{k} + \cdots + \alpha_{0} \cdot e^{0}
= 

$$

\section{ZK4Z2K paper}

\subsection{MPC protocol}
the MPC protocol's purpose is to verify the evaluation of given circuit C is honest-In essence, this protocol does not compute the circuit C, but only checks that the values given by the input party are consistent with an honest evaluation of C.

Linear gates are calculated locally and non-linear are checked using the subprotcols. Linear calculations are preserved using the t-privacy?

\subsection{MPCitH}
using Ishai's compilation method. The prover executes multiple parties in its head, and the prover provides the answer to any query to the hint oracle $O_H$(a hint oracle which provides the parties with a sharing of an arbitrary secret value from the input party).

\subsection{sacrificed check}
the sacrificed check is to check whether the prover is cheating with the multiplication result $z$ or the hint oracle $O_H$. The main idea is to use one input gate $y$, generate secret values to check if $a y = c$ to check if the secret values are correct, i.e. the hint oracle does not cheat. Then it randomizes a vector using $O_R$ and sacrifices a secret value to mask the other multiplication input gate $x$

\subsection{inner multiplication check}

see Limbo where notations are less confusing

\section{fast embedding}

TODO: write down the in-depth description of the method by robin
\section{Random Oracle}
We are using the SHA-3 function family to construct a random oracle, which works as follows:

\begin{itemize}
\item fix a seed
\item initialize a counter
\item $H(seed \|\| counter)$
\item $count += 1$
\end{itemize}

or 
(currently I am using, but I dont see why I use a PRNG in-between, I think i can use the method above instead)

call a PRNG api(its own inner state or just refresh seed with $seed \|\| counter$) to calculate H(PRNG(seed with sufficient entropy)).


\section{LSSS}




%+Bibliography
\begin{thebibliography}{99}
\bibitem{Label1} ...
\bibitem{Label2} ...
\end{thebibliography}
%-Bibliography

\end{document}


